% A LaTeX template for MSc Thesis submissions to 
% Politecnico di Milano (PoliMi) - School of Industrial and Information Engineering
%
% S. Bonetti, A. Gruttadauria, G. Mescolini, A. Zingaro
% e-mail: template-tesi-ingind@polimi.it
%
% Last Revision: October 2021
%
% Copyright 2021 Politecnico di Milano, Italy. NC-BY

\documentclass{Configuration_Files/PoliMi3i_thesis}

%------------------------------------------------------------------------------
%	REQUIRED PACKAGES AND  CONFIGURATIONS
%------------------------------------------------------------------------------

% CONFIGURATIONS
\usepackage{parskip} % For paragraph layout
\usepackage{setspace} % For using single or double spacing
\usepackage{emptypage} % To insert empty pages
\usepackage{multicol} % To write in multiple columns (executive summary)
\setlength\columnsep{15pt} % Column separation in executive summary
\setlength\parindent{0pt} % Indentation
\raggedbottom  

% PACKAGES FOR TITLES
\usepackage{titlesec}
% \titlespacing{\section}{left spacing}{before spacing}{after spacing}
\titlespacing{\section}{0pt}{3.3ex}{2ex}
\titlespacing{\subsection}{0pt}{3.3ex}{1.65ex}
\titlespacing{\subsubsection}{0pt}{3.3ex}{1ex}
\usepackage{color}

% PACKAGES FOR LANGUAGE AND FONT
\usepackage[english]{babel} % The document is in English  
\usepackage[utf8]{inputenc} % UTF8 encoding
\usepackage[T1]{fontenc} % Font encoding
\usepackage[11pt]{moresize} % Big fonts

% PACKAGES FOR IMAGES
\usepackage{graphicx}
\usepackage{transparent} % Enables transparent images
\usepackage{eso-pic} % For the background picture on the title page
\usepackage{subfig} % Numbered and caption subfigures using \subfloat.
\usepackage{tikz} % A package for high-quality hand-made figures.
\usetikzlibrary{}
\graphicspath{{./Images/}} % Directory of the images
\usepackage{caption} % Coloured captions
\usepackage{xcolor} % Coloured captions
\usepackage{amsthm,thmtools,xcolor} % Coloured "Theorem"
\usepackage{float}

% STANDARD MATH PACKAGES
\usepackage{amsmath}
\usepackage{amsthm}
\usepackage{amssymb}
\usepackage{amsfonts}
\usepackage{bm}
\usepackage[overload]{empheq} % For braced-style systems of equations.
\usepackage{fix-cm} % To override original LaTeX restrictions on sizes

% PACKAGES FOR TABLES
\usepackage{tabularx}
\usepackage{longtable} % Tables that can span several pages
\usepackage{colortbl}

% PACKAGES FOR ALGORITHMS (PSEUDO-CODE)
\usepackage{algorithm}
\usepackage{algorithmic}

% PACKAGES FOR REFERENCES & BIBLIOGRAPHY
\usepackage[colorlinks=true,linkcolor=black,anchorcolor=black,citecolor=black,filecolor=black,menucolor=black,runcolor=black,urlcolor=black]{hyperref} % Adds clickable links at references
\usepackage{cleveref}
\usepackage[square, numbers, sort&compress]{natbib} % Square brackets, citing references with numbers, citations sorted by appearance in the text and compressed
\bibliographystyle{abbrvnat} % You may use a different style adapted to your field

% OTHER PACKAGES
\usepackage{pdfpages} % To include a pdf file
\usepackage{afterpage}
\usepackage{lipsum} % DUMMY PACKAGE
\usepackage{fancyhdr} % For the headers
\fancyhf{}

\usepackage{graphicx}
\usepackage{tocloft}  
\usepackage{titling}  
\usepackage{booktabs} 
\usepackage{tabularx}  
\usepackage{geometry} 
\usepackage{longtable}
\usepackage{multirow}
\usepackage{todonotes}
\usepackage{listings}
\usepackage{xcolor}
\usepackage{amsmath} 
\usepackage[colorlinks=true, linkcolor=black, urlcolor=black, citecolor=black, filecolor=black]{hyperref}

%PlantUml package
\usepackage{plantuml}


% Define colors for custom Alloy highlighting
\definecolor{commentgreen}{rgb}{0,0.6,0}
\definecolor{keywordblue}{rgb}{0.13,0.13,1}
\definecolor{stringred}{rgb}{0.8,0,0}
\definecolor{numberpurple}{rgb}{0.58,0,0.83}

% Define Alloy language syntax highlighting
\lstdefinelanguage{Alloy}{
    keywords={sig, extends, fact, pred, fun, run, assert, check, some, all, no, one, lone, abstract, disj, module, open, set},
    keywordstyle=\color{keywordblue}\bfseries,
    comment=[l]{//},              % comment style
    commentstyle=\color{commentgreen},
    stringstyle=\color{stringred},
    morestring=[b]",              % strings are in quotes
    sensitive=true,               % case sensitive
    morekeywords={in, and, or, not, implies, iff}, % Alloy-specific keywords
    morecomment=[s]{/*}{*/},      % block comment style
    literate={0}{{\textcolor{numberpurple}{0}}}{1}%
             {1}{{\textcolor{numberpurple}{1}}}{1}%
             {2}{{\textcolor{numberpurple}{2}}}{1}%
             {3}{{\textcolor{numberpurple}{3}}}{1}%
             {4}{{\textcolor{numberpurple}{4}}}{1}%
             {5}{{\textcolor{numberpurple}{5}}}{1}%
             {6}{{\textcolor{numberpurple}{6}}}{1}%
             {7}{{\textcolor{numberpurple}{7}}}{1}%
             {8}{{\textcolor{numberpurple}{8}}}{1}%
             {9}{{\textcolor{numberpurple}{9}}}{1}%
}

\lstset{
    language=Alloy,
    basicstyle=\ttfamily\footnotesize,
    numbers=left,
    numberstyle=\tiny\color{gray},
    stepnumber=1,
    breaklines=true,
    showstringspaces=false,
    frame=single,
    caption=Alloy Example,
    label=code:alloy_example,
}

\setlength{\cftbeforesecskip}{10pt}


% Input of configuration file. Do not change config.tex file unless you really know what you are doing. 
% Define blue color typical of polimi
\definecolor{bluepoli}{cmyk}{0.4,0.1,0,0.4}

% Custom theorem environments
\declaretheoremstyle[
  headfont=\color{bluepoli}\normalfont\bfseries,
  bodyfont=\color{black}\normalfont\itshape,
]{colored}

% Set-up caption colors
\captionsetup[figure]{labelfont={color=bluepoli}} % Set colour of the captions
\captionsetup[table]{labelfont={color=bluepoli}} % Set colour of the captions
\captionsetup[algorithm]{labelfont={color=bluepoli}} % Set colour of the captions

\theoremstyle{colored}
\newtheorem{theorem}{Theorem}[chapter]
\newtheorem{proposition}{Proposition}[chapter]

% Enhances the features of the standard "table" and "tabular" environments.
\newcommand\T{\rule{0pt}{2.6ex}}
\newcommand\B{\rule[-1.2ex]{0pt}{0pt}}

% Pseudo-code algorithm descriptions.
\newcounter{algsubstate}
\renewcommand{\thealgsubstate}{\alph{algsubstate}}
\newenvironment{algsubstates}
  {\setcounter{algsubstate}{0}%
   \renewcommand{\STATE}{%
     \stepcounter{algsubstate}%
     \Statex {\small\thealgsubstate:}\space}}
  {}

% New font size
\newcommand\numfontsize{\@setfontsize\Huge{200}{60}}

% Title format: chapter
\titleformat{\chapter}[hang]{
\fontsize{50}{20}\selectfont\bfseries\filright}{\textcolor{bluepoli} \thechapter\hsp\hspace{2mm}\textcolor{bluepoli}{|   }\hsp}{0pt}{\huge\bfseries \textcolor{bluepoli}
}

% Title format: section
\titleformat{\section}
{\color{bluepoli}\normalfont\Large\bfseries}
{\color{bluepoli}\thesection.}{1em}{}

% Title format: subsection
\titleformat{\subsection}
{\color{bluepoli}\normalfont\large\bfseries}
{\color{bluepoli}\thesubsection.}{1em}{}

% Title format: subsubsection
\titleformat{\subsubsection}
{\color{bluepoli}\normalfont\large\bfseries}
{\color{bluepoli}\thesubsubsection.}{1em}{}

% Shortening for setting no horizontal-spacing
\newcommand{\hsp}{\hspace{0pt}}

\makeatletter
% Renewcommand: cleardoublepage including the background pic
\renewcommand*\cleardoublepage{%
  \clearpage\if@twoside\ifodd\c@page\else
  \null
  \AddToShipoutPicture*{\BackgroundPic}
  \thispagestyle{empty}%
  \newpage
  \if@twocolumn\hbox{}\newpage\fi\fi\fi}
\makeatother

%For correctly numbering algorithms
\numberwithin{algorithm}{chapter}


%----------------------------------------------------------------------------
%	NEW COMMANDS DEFINED
%----------------------------------------------------------------------------

% EXAMPLES OF NEW COMMANDS
\newcommand{\bea}{\begin{eqnarray}} % Shortcut for equation arrays
\newcommand{\eea}{\end{eqnarray}}
\newcommand{\e}[1]{\times 10^{#1}}  % Powers of 10 notation

%----------------------------------------------------------------------------
%	ADD YOUR PACKAGES (be careful of package interaction)
%----------------------------------------------------------------------------

%----------------------------------------------------------------------------
%	ADD YOUR DEFINITIONS AND COMMANDS (be careful of existing commands)
%----------------------------------------------------------------------------

%----------------------------------------------------------------------------
%	BEGIN OF YOUR DOCUMENT
%----------------------------------------------------------------------------

\begin{document}

\fancypagestyle{plain}{%
\fancyhf{} % Clear all header and footer fields
\fancyhead[RO,RE]{\thepage} %RO=right odd, RE=right even
\renewcommand{\headrulewidth}{0pt}
\renewcommand{\footrulewidth}{0pt}}

%----------------------------------------------------------------------------
%	TITLE PAGE
%----------------------------------------------------------------------------

\pagestyle{empty} % No page numbers
\frontmatter % Use roman page numbering style (i, ii, iii, iv...) for the preamble pages

\puttitle{
    title=ITD, % Title of the thesis
    nameA=Giovanni Vaccarino, % Author Name and Surname
    nameB=Vittorio Palladino, % Author Name and Surname
    nameC=Nicolò Vacis, % Author Name and Surname
    course=Computer Science and Engineering, % Study Programme (in Italian)
    academicyear={2024{-}25},  % Academic Year
} % These info will be put into your Title page 

%----------------------------------------------------------------------------
%	LIST OF CONTENTS/FIGURES/TABLES/SYMBOLS
%----------------------------------------------------------------------------

% TABLE OF CONTENTS
\thispagestyle{empty}
\tableofcontents % Table of contents 
\thispagestyle{empty}
\cleardoublepage

%-------------------------------------------------------------------------
%	THESIS MAIN TEXT
%-------------------------------------------------------------------------
% In the main text of your thesis you can write the chapters in two different ways:
%
%(1) As presented in this template you can write:
%    \chapter{Title of the chapter}
%    *body of the chapter*
%
%(2) You can write your chapter in a separated .tex file and then include it in the main file with the following command:
%    \chapter{Title of the chapter}
%    \input{chapter_file.tex}
%
% Especially for long thesis, we recommend you the second option.

\addtocontents{toc}{\vspace{2em}} % Add a gap in the Contents, for aesthetics
\mainmatter % Begin numeric (1,2,3...) page numbering

% --------------------------------------------------------------------------
% NUMBERED CHAPTERS % Regular chapters following
% --------------------------------------------------------------------------

\chapter{Introduction}
\section{Scope}

The S\&C platform is a web-based solution intended to support:
\begin{itemize}
    \item \textbf{Students} in finding internships suited to their skills and aspirations, allowing them to submit CVs and receive tailored recommendations.
    \item \textbf{Companies} in listing internship opportunities and identifying candidates based on skills and profiles.
    \item \textbf{Universities} in overseeing internships, tracking progress, addressing complaints, and ensuring the quality of internships.
\end{itemize}
The platform enables interaction among these groups, handling tasks like matching candidates with internships, interview scheduling, feedback collection, and assessing internship effectiveness. The platform is built to scale across universities, supporting thousands of users.

\section{Definitions, Acronyms, Abbreviations}

\subsection{Definitions}
\begin{itemize}
    \item \textbf{Slug}: A human-readable and URL-friendly string (typically lowercase ASCII characters) that uniquely identifies a specific resource. It is commonly used in URLs but may also serve as an identifier for resources, like internship listings or profiles, within the S\&C platform.
    \item \textbf{S\&C Repository Slug}: A slug in the format `<repository owner>/<repository name>` that uniquely identifies a project repository on the S\&C platform or external GitHub repositories associated with student or company projects.
\end{itemize}

\subsection{Acronyms}
\begin{itemize}
    \item \textbf{S\&C}: Students\&Companies platform
    \item \textbf{API}: Application Programming Interface
    \item \textbf{CV}: Curriculum Vitae
    \item \textbf{GDPR}: General Data Protection Regulation
    \item \textbf{GH}: GitHub
    \item \textbf{SSO}: Single Sign-On
    \item \textbf{UUID}: Universal Unique Identifier
    \item \textbf{DB}: Database
    \item \textbf{DBMS}: Database Management System
    \item \textbf{RPC}: Remote Procedure Call
    \item \textbf{REST}: Representational State Transfer
    \item \textbf{SPA}: Single Page Application
    \item \textbf{CDN}: Content Delivery Network
\end{itemize}

\subsection{Abbreviations}
\begin{itemize}
    \item \textbf{e.g.}: For example
    \item \textbf{repo}: Repository
    \item \textbf{ID}: Identifier
\end{itemize}

\section{Revision History}
\begin{table}[h!]
\centering
\begin{tabular}{|c|c|l|}
\hline
\textbf{Version} & \textbf{Date} & \textbf{Description} \\ \hline
1.0 & January 7, 2024 & Initial release \\ \hline
\end{tabular}
\end{table}


\section{Reference Documents}
\begin{itemize}
    \item Specification document: "Assignment RDD AY 2023-2024"
    \item UML official specification: \url{https://www.omg.org/spec/UML/}
    \item Requirements Analysis and Specification Document: "RASD"
\end{itemize}

\section{Document Structure}
The document is organized into the following sections:
\begin{itemize}
    \item \textbf{Section 1: Introduction} - Provides an overview of the problem and the system's scope. It also includes definitions, acronyms, abbreviations, and a revision history to track document versions and modifications.
    \item \textbf{Section 2: Architectural Design} - Describes the system architecture, starting with a high-level overview of design choices, followed by detailed component and deployment views, including server, client, and data components (DB schemas). This section also presents sequence diagrams to illustrate the system’s runtime view and includes information on component interfaces, design styles, and architectural patterns.
    \item \textbf{Section 3: User Interface Design} - Outlines the design of the user interface, providing an overview of the platform’s visual layout and user interactions.
    \item \textbf{Section 4: Requirements Traceability} - Maps the requirements from the RASD to corresponding design elements in this document.
    \item \textbf{Section 5: Implementation, Integration, and Test Plan} - Details the implementation order of the system’s subcomponents, integration steps, and testing procedures to ensure system reliability.
\end{itemize}



\chapter{Implemented Requirements}
\noindent\textbf{[R1]: The S\&C shall allow users to log in to their accounts using their registered email and password.} \\
\textit{This requirement has been implemented.} Users can successfully log in to their accounts with their email and password.

\noindent\rule{16cm}{0.4pt}

\noindent\textbf{[R2]: The S\&C shall allow users to reset their password via a “Forgot Password” feature, with verification sent to their registered email.} \\
\textit{This requirement has been implemented.} Users can reset their password via the “Forgot Password” feature, and the system sends a verification email.

\noindent\rule{16cm}{0.4pt}

\noindent\textbf{[R3]: The S\&C shall allow users to log out securely from their accounts.} \\
\textit{This requirement has been implemented.} Users can securely log out from their accounts without issues.

\noindent\rule{16cm}{0.4pt}

\noindent\textbf{[R4]: The S\&C shall allow users to register on the platform.} \\
\textit{This requirement has been implemented.} Users can create accounts on the platform.

\noindent\rule{16cm}{0.4pt}

\noindent\textbf{[R5]: The S\&C shall allow students to edit profiles and upload CV.} \\
\textit{This requirement has been implemented.} Students can edit their profiles and upload their CVs successfully.

\noindent\rule{16cm}{0.4pt}

\noindent\textbf{[R6]: The S\&C shall allow students to edit personal details, skills, and interests in their profiles.} \\
\textit{This requirement has been implemented.} Students can update personal details, skills, and interests in their profiles.

\noindent\rule{16cm}{0.4pt}

\noindent\textbf{[R7]: The S\&C shall allow students to submit assessments, including answering open-ended and multiple-choice questions.} \\
\textit{This requirement has been implemented.} Students can submit assessments with both open-ended and multiple-choice questions.

\noindent\rule{16cm}{0.4pt}

\noindent\textbf{[R8]: The S\&C shall provide students with an internship search tool that includes filtering options for location, job title, and publication date.} \\
\textit{This requirement has been implemented.} Students can search for internships with filters for location, job title, and publication date.

\noindent\rule{16cm}{0.4pt}

\noindent\textbf{[R9]: The S\&C shall allow students to get tailored internship suggestions based on their interests and skills.} \\
\textit{This requirement has been implemented.} Students receive tailored internship suggestions based on their specified interests and skills.

\noindent\rule{16cm}{0.4pt}

\noindent\textbf{[R10]: The S\&C shall allow the recommendation engine to improve over time based on feedback from students and companies.} \\
\textit{This requirement has been implemented.} The recommendation engine improves over time using feedback from students and companies.

\noindent\rule{16cm}{0.4pt}

\noindent\textbf{[R11]: The S\&C shall allow the recommendation engine to improve over time based on feedback from students and companies.} \\
\textit{This requirement has been implemented.} The recommendation engine improves over time using feedback from students and companies.

\noindent\rule{16cm}{0.4pt}

\noindent\textbf{[R12]: The S\&C shall allow real-time tracking of application statuses, showing students the progress of applications at each stage.} \\
\textit{This requirement has been implemented.} Students can track their application statuses in real-time, viewing progress at each stage.

\noindent\rule{16cm}{0.4pt}

\noindent\textbf{[R13]: The S\&C shall allow companies to view submitted applications.} \\
\textit{This requirement has been implemented.} Companies can access and view the applications submitted by students.

\noindent\rule{16cm}{0.4pt}

\noindent\textbf{[R14]: The S\&C shall allow companies to accept or reject candidates directly from the submitted applications.} \\
\textit{This requirement has been implemented.} Companies can accept or reject candidates directly from the submitted applications.

\noindent\rule{16cm}{0.4pt}

\noindent\textbf{[R15]: The S\&C shall allow companies to create internship postings, specifying job roles and required skills.} \\
\textit{This requirement has been implemented.} Companies can create detailed internship postings specifying job roles and required skills.

\noindent\rule{16cm}{0.4pt}

\noindent\textbf{[R16]: The S\&C shall allow companies to edit existing internship postings after creation.} \\
\textit{This requirement has} \textbf{not} \textit{been implemented.}Companies can edit their existing internship postings after they have been created.

\noindent\rule{16cm}{0.4pt}

\noindent\textbf{[R17]: The S\&C shall allow companies to have a dashboard to review candidate applications.} \\
\textit{This requirement has been implemented.} Companies can use the dashboard to review.

\noindent\rule{16cm}{0.4pt}

\noindent\textbf{[R18]: The S\&C shall allow companies access to a comprehensive page to track applications status.} \\
\textit{This requirement has been implemented.} Companies have access to a detailed page for tracking application statuses.

\noindent\rule{16cm}{0.4pt}

\noindent\textbf{[R20]: The S\&C shall allow companies to access a ``Potential Candidates'' feature to view potential matches based on student profiles.} \\
\textit{This requirement has been implemented.} Companies can use the ``Potential Candidates'' feature to find student matches.

\noindent\rule{16cm}{0.4pt}

\noindent\textbf{[R21]: The S\&C shall allow both students and companies to provide feedback, covering overall satisfaction.} \\
\textit{This requirement has been implemented.} Both students and companies can provide overall satisfaction feedback.

\noindent\rule{16cm}{0.4pt}

\noindent\textbf{[R22]: The S\&C shall allow companies to view feedback provided for candidates from previous internships.} \\
\textit{This requirement has been implemented.} Companies can access feedback for candidates from previous internships.

\noindent\rule{16cm}{0.4pt}

\noindent\textbf{[R23]: The S\&C shall allow students to view feedback provided for companies from previous interns.} \\
\textit{This requirement has been implemented.} Students can view feedback for companies given by previous interns.

\noindent\rule{16cm}{0.4pt}

\noindent\textbf{[R25]: The S\&C shall allow a reporting feature for users to report inappropriate or irrelevant internship environments, ensuring a safe and professional place.} \\
\textit{This requirement has been implemented.} Users can report inappropriate or irrelevant internship environments.

\noindent\rule{16cm}{0.4pt}



\chapter{Adopted Frameworks}
\section{Technology Stack Selection}

In this section, the rationale for choosing the specific frameworks and languages for the backend and frontend development of the job-boarding platform is discussed.

\subsection{Backend: ASP.NET Core with C\#}
The backend framework chosen for the application is ASP.NET Core, utilizing the C\# programming language. The primary reasons for this choice are outlined below:

\begin{itemize}
    \item \textbf{Cross-Platform Support:} ASP.NET Core is a modern, cross-platform framework that enables the development of web applications on Windows, Linux, and macOS, ensuring deployment flexibility.
    
    \item \textbf{Performance:} ASP.NET Core is known for its exceptional performance and speed, which are critical for handling concurrent requests in a scalable job-boarding platform.
    
    \item \textbf{Robust Ecosystem:} The .NET ecosystem offers a wide range of libraries, tools, and integrations, which accelerates development and reduces the need for third-party dependencies.
    
    \item \textbf{Built-In Pattern Implementations:} ASP.NET Core provides easy access to widely adopted software patterns such as Dependency Injection (DI) and Mediator, making it straightforward to implement clean architecture principles and maintain a modular codebase.
    
    \item \textbf{Security:} ASP.NET Core includes built-in features for managing authentication, authorization, and data protection, making it easier to build secure applications.
    
    \item \textbf{Community and Support:} ASP.NET Core has a large and active community, ensuring access to extensive documentation, tutorials, and third-party tools.
\end{itemize}

\textbf{Disadvantages:}
\begin{itemize}
    \item \textbf{Learning Curve:} For developers unfamiliar with .NET Core, the framework can have a steep learning curve, especially when adopting patterns such as DI and Mediator.
    \item \textbf{Overhead for Small Projects:} The robust ecosystem and extensive features may introduce unnecessary overhead for simpler applications.
\end{itemize}

\subsection{Frontend: React with TypeScript}
For the frontend, the chosen framework is React, combined with TypeScript. The reasons for this selection are as follows:

\begin{itemize}
    \item \textbf{Component-Based Architecture:} React's component-based design simplifies the creation of reusable and modular UI components. This feature of React has been crucial during the frontend development.
    
    \item \textbf{Rich Ecosystem:} React has a vast ecosystem of libraries and tools that we've used, such as React Router for navigation and state management solutions like Redux. Moreover, for the design, we've chosen to use the MUI library.

    \item \textbf{Dependency Injection Support:} Libraries like \texttt{Inversify} made it easier to implement the Dependency Injection (DI) pattern on the frontend, promoting clean architecture principles and maintaining consistency between the frontend and backend.

    \item \textbf{Type Safety with TypeScript:} TypeScript enhances JavaScript by adding static typing, reducing runtime errors and improving code maintainability in large applications.
    
    \item \textbf{Performance:} React's virtual DOM efficiently updates the user interface, ensuring fast rendering and a responsive user experience.

    \item \textbf{Popularity and Community Support:} React is one of the most popular frontend frameworks, with a strong community and extensive resources for learning and troubleshooting.
\end{itemize}

\textbf{Disadvantages:}
\begin{itemize}
    \item \textbf{Boilerplate Code:} While React provides flexibility, managing state and other common tasks often requires additional libraries and setup, leading to boilerplate code.
    \item \textbf{Tooling Complexity:} Setting up a React project with TypeScript, state management, and DI libraries can be complex and time-consuming for developers new to the ecosystem.
\end{itemize}

\subsection{Overall Justification and Limitations}
The combination of ASP.NET Core with C\# for the backend and React with TypeScript for the frontend provides a well-rounded, modern technology stack. 

\textbf{Limitations:}
\begin{itemize}
    \item The initial learning curve for both ASP.NET Core and React with TypeScript may increase the onboarding time for new developers.
    \item Integration between the frontend and backend may require careful coordination, particularly when implementing DI patterns and managing shared state.
\end{itemize}


\chapter{Source Code Structure}
\section{Backend}
This section provides an overview of the source code structure for the backend of the platform. The code is organized into three main layers: \textbf{Business}, \textbf{Data}, and \textbf{Service}, each serving a distinct purpose. Additionally, a \textbf{Shared} directory contains common utilities and resources.

\subsection{Business Layer}
The \textbf{Business} layer is responsible for handling the business logic of the application. This layer is structured to follow the \textbf{Command Query Responsibility Segregation (CQRS)} pattern. The organization is as follows:

\begin{itemize}
    \item For each section of the application (e.g., \textit{Authentication}, \textit{Internship}, \textit{Student}, etc.), there is a dedicated folder.
    \item Within each section folder, the relevant \textbf{use cases} are implemented. 
    \item Each use case folder contains:
    \begin{itemize}
        \item The \textbf{Command} or \textbf{Query}: These define the data and parameters required for executing the use case, and also its return value.
        \item The corresponding \textbf{Use Case}: This contains the logic for processing the command or query.
    \end{itemize}
\end{itemize}

\subsection{Data Layer}
The \textbf{Data} layer manages all database-related configurations and entity definitions. It is organized as follows:

\begin{itemize}
    \item A file for the \textbf{database setup}, which includes configuration settings, database initialization. This file also handles \textbf{entity relations} and cascading policies, ensuring proper database constraints and relationships.
    \item A folder named \texttt{Entities}, which contains the definitions of all the \textbf{entities} in the application. Each entity corresponds to a table in the database and includes properties and navigation relationships. Each entity extends the \textbf{EntityBase}, that represents the base properties of an entity: an \textbf{autogenerated ID}, a \textbf{createdDate} and an \textbf{updatedDate}
\end{itemize}

\subsection{Service Layer}
The \textbf{Service} layer serves as the interaction point between the application and the external users or systems. This layer includes:

\begin{itemize}
    \item A \texttt{Contracts} folder containing the \textbf{DTOs (Data Transfer Objects)}, which are organized into subfolders corresponding to each section (e.g., \textit{Authentication}, \textit{Internship}, etc.).
    \item A \texttt{Controllers} folder that includes all the \textbf{controllers} responsible for processing HTTP requests and returning responses.
    \item A \texttt{Middlewares} folder that contains any middlewares or policies(for authorization)
\end{itemize}

\subsection{Shared Folder}
Outside the main three layers, a \textbf{Shared} folder contains resources and utilities that are reused across the application. This includes:

\begin{itemize}
    \item A folder named \texttt{Enums}, which holds all the \textbf{enumerations} used throughout the application.
    \item A folder named \texttt{Services}, which contains shared services such as:
    \begin{itemize}
        \item The \textbf{Storage Service} for handling file uploads and storage operations.
        \item The \textbf{Email Service} for sending emails.
        \item The \textbf{Background Matching Job Service}, a folder that contains many files for handling the matching background job.
    \end{itemize}
\end{itemize}


\section{Frontend}
This section provides an overview of the source code structure for the frontend of the platform. The code has many files in the root folder, that handle for example routing or theming, but the core of the code is organized into several directories:

\subsection{Pages Folder}
The \texttt{pages} folder contains all the \textbf{pages} of the application. Each page represents a distinct view or route, such as the login page, home page, or internship details page.

\subsection{Components Folder}
The \texttt{components} folder contains all the \textbf{reusable UI components} in the application. The components are further categorized into subfolders based on their type, such as:
\begin{itemize}
    \item \texttt{layout}: Components related to the layout, such as headers, footers, and sidebars.
    \item \texttt{buttons}: Custom button components used across the application.
    \item \texttt{list-items}: Components for rendering lists and list items.
    \item Other types of reusable components.
\end{itemize}

\subsection{Hooks Folder}
The \texttt{hooks} folder contains all the \textbf{custom hooks} created for the application. These hooks are often wrappers that encapsulate reusable logic.

\subsection{Models Folder}
The \texttt{models} folder is used to define all the \textbf{interfaces and types} for the application's data models. For example:
\begin{itemize}
    \item \texttt{auth}: Contains interfaces/types related to authentication, such as login payloads or user models.
    \item \texttt{application}: Contains interfaces/types for internship applications.
\end{itemize}
Since we are working in TypeScript, this folder is crucial to have a specific point where to define our types.

\subsection{Core Folder}
The \texttt{core} folder contains essential services and utilities that form the core infrastructure of the frontend. It is structured as follows:

\begin{itemize}
    \item \texttt{config}: This folder contains files for setting up and managing the \textbf{configuration service}.
    \item \texttt{ioc}: This folder is dedicated to \textbf{inversion of control (IoC)} and is set up using \texttt{Inversify}. It manages dependency injection across the application.
    \item \texttt{store}: This folder is used for \textbf{Redux state management}. It includes:
    \begin{itemize}
        \item Configuration files for setting up the Redux store.
        \item A \texttt{slices} folder, where each slice represents a distinct piece of application state, such as authentication state or internship data.
    \end{itemize}
    \item \texttt{API}: This folder handles API-related logic. It is further divided into:
    \begin{itemize}
        \item \texttt{setup-api}: Contains files for setting up the API infrastructure, including factories and decorators.
        \item Subfolders for each API (e.g., \texttt{auth}, \texttt{internship}), where each subfolder includes:
        \begin{itemize}
            \item An \textbf{interface} defining the API contract(e.g IInternshipApi).
            \item One or more \textbf{implementations} of the API(e.g InternshipApi).
        \end{itemize}
    \end{itemize}
\end{itemize}

\section{Patterns and Architectural Choices}

This section explains the design patterns and architectural principles adopted across both the backend and frontend, where we have always tried to follow the \textbf{SOLID} principles.

\subsection{Backend}

\begin{itemize}
    \item \textbf{Dependency Injection (DI)}: Dependency Injection is used to manage dependencies, promoting loose coupling between components, simplifying testing.
    
    \item \textbf{Command Query Responsibility Segregation (CQRS)}: The CQRS pattern is used in the \textbf{Business Layer}, where commands and queries are handled separately. This allows for better scalability and separation of responsibilities.

    \item \textbf{Mediator}: The Mediator pattern is used to coordinate communication between different components, especially in the context of processing use cases.

    \item \textbf{Mapper}: A mapper is employed to handle conversions between \textbf{entities} and \textbf{DTOs (Data Transfer Objects)}, ensuring that the data structure used in the service layer and data layer remains consistent.
\end{itemize}

\subsection{Frontend}
\begin{itemize}
    \item \textbf{Dependency Injection (DI)}: Dependency Injection is implemented using \texttt{Inversify} to manage services and dependencies across the application.

    \item \textbf{Builder}: The builder pattern is utilized in the setup for the configuration service.

    \item \textbf{Factory}: Factories are used to create instances of various services, including API setup.

    \item \textbf{Decorator}: Decorators are applied for instance to API setup, in order to decorate the HttpClient to handle RefreshToken procedures.

    \item \textbf{State Management (Redux)}: The \textbf{store} directory utilizes Redux for managing global application state. Redux slices represent different parts of the state, such as authentication.
\end{itemize}



\chapter{Testing}
As detailed in the Design Document (Chapter 5), we adopted a bottom-up implementation approach during the development of the platform. For each section of the platform, including Authentication, Internship management, and more, we implemented the corresponding Use Cases and conducted Unit Tests in parallel. Specifically, we utilized the \texttt{xUnit} testing framework and developed an \texttt{IsolatedUnitTests} class. This class provided an isolated environment for creating unit tests, including all necessary dependencies and mock objects. 

To illustrate, for the \texttt{RegisterUseCase}, we conducted the following key unit tests:
\begin{itemize}
    \item Successful registration of a student.
    \item Successful registration of a company.
    \item Handling of already registered email addresses.
    \item Validation of password strength.
\end{itemize}

We aimed for a line code coverage of at least 97\% in each section, ensuring that every line of code functioned as intended. Performing these unit tests during the development of each section was critical for maintaining consistent functionality over time, essentially forming a continuous integration (CI) workflow.

After implementing and unit testing all sections of the platform, we moved to integration testing. This phase was designed to verify that the individual building blocks of the platform worked in the right way together. Integration testing included comprehensive scenarios such as:
\begin{itemize}
    \item For students: registration, profile editing, and applying to internships.
    \item For companies: registration, profile editing, and internship creation.
\end{itemize}

In addition to verifying correct functionality and database effects, we ensured that HTTP requests returned the correct response codes for each interaction. As with unit testing, \texttt{xUnit} was used for integration tests.

The final phase of testing involved End-to-End (E2E) testing, conducted using \texttt{Cypress}, to simulate real-world scenarios and ensure the platform functioned correctly from the user's perspective. E2E testing included:
\begin{itemize}
    \item Verification of the user interface: ensuring it rendered correctly across different browsers and devices.
    \item Testing critical user workflows: such as student registration, internship application, and company creation of internships.
    \item Validation of edge cases: such as handling invalid user inputs or ensuring robust error recovery mechanisms.
    \item Performance testing: measuring response times for key user actions to ensure the platform's responsiveness.
\end{itemize}

The E2E tests not only validated the correctness of the platform but also ensured a smooth and intuitive user experience. \texttt{Cypress} provided an efficient framework for automating these scenarios, enabling us to identify and address issues early. This comprehensive testing strategy ensured that our platform met the highest standards of reliability and usability.


\subsection{Static Analysis}
In addition to unit, integration, and end-to-end testing, we also performed static analysis on the codebase to ensure code quality at static time. Static analysis has been done using linters which helped us identify potential issues such as syntax errors, unused variables, and violations of coding standards. For example, \texttt{ESLint} helped ensure TypeScript type safety and React best practices. Additionally, we employed code formatters such as \texttt{Prettier} to enforce consistent code formatting across the project. While formatters do not analyze code behavior, their contribution to maintaining readability and consistency aligns with the goals of static analysis.


\chapter{Installation Instructions}
This chapter provides detailed instructions for setting up and running the platform locally, including both the backend (written in ASP.NET Core) and the frontend (developed in React with TypeScript). Follow these steps carefully to ensure proper configuration and execution.

\section{Prerequisites}

Before proceeding with the installation, ensure that the following prerequisites are met:
\begin{itemize}
    \item \textbf{Backend Requirements}:
    \begin{itemize}
        \item .NET SDK (version 8.0 or later)
        \item A code editor (e.g., Rider or Visual Studio Code)
    \end{itemize}
    \item \textbf{Frontend Requirements}:
    \begin{itemize}
        \item Node.js (version 18.x or later) and npm (version 9.x or later)
        \item A code editor (e.g., Visual Studio Code)
    \end{itemize}
\end{itemize}

\section{Setting up}

\begin{enumerate}
    \item Copy the folder in the zip file named "VaccarinoPalladinoVacis" wherever you prefer on your device
\end{enumerate}

\subsection{Setting Up the Backend}

The backend is implemented as an ASP.NET Core application. Follow these steps to set it up:
\begin{enumerate}
    \item Consider the backend folder(assuming you're in the "VaccarinoPalladinoVacis" folder:
    \begin{verbatim}
    cd SC/backend
    \end{verbatim}

    \item \textbf{Configure the environment variables}.\\\\ In the zip file you can find a directory called env-files that contains all the necessary file to copy. In particular for the backend you will copy:
    \begin{itemize}
        \item An \texttt{appsettings.json} file containing the required configuration (e.g., database connection string, API keys, etc.) Place this file in the root of the backend project directory.
        \item An \texttt{appsettings.Development.json} file containing the required configuration for the development settings. Place this file in the root of the backend project directory.
        \item A \texttt{launchSettings.json} file containing the required settings for launching. Place this file into a directory called \texttt{"Properties"}, that must be placed in the root of the backend project directory.
    \end{itemize}

    \item Restore dependencies:
    \begin{verbatim}
    dotnet restore
    \end{verbatim}

    \item Run the application locally:
    \begin{verbatim}
    dotnet run
    \end{verbatim}
    The backend will typically run on \texttt{https://localhost/api:5000} by default. \textbf{Ensure the port is 5000}
\end{enumerate}

\subsection{Setting Up the Frontend}

The frontend is a React application written in TypeScript. Follow these steps to set it up:
\begin{enumerate}
    \item Consider the frontend folder(assuming you're in the "VaccarinoPalladinoVacis" folder:
    \begin{verbatim}
    cd SC/backend
    \end{verbatim}

    \item Install dependencies:
    \begin{verbatim}
    npm install
    \end{verbatim}

    \item Run the application locally:
    \begin{verbatim}
    npm start
    \end{verbatim}
    The frontend will typically be available at \texttt{http://localhost:5173}. \textbf{Ensure the running port is 5173, in order to don't face CORS policies issues}.
\end{enumerate}

\section{Running the Platform Locally}

Once both the backend and frontend are set up, follow these steps to run the entire platform:
\begin{enumerate}
    \item Start the backend (if not already started) by running \texttt{dotnet run}.
    \item Start the frontend (if not already started) by running \texttt{npm start}.
    \item Open a browser and navigate to \texttt{http://localhost:5173} to access the platform.
\end{enumerate}

\section{Troubleshooting}

\begin{itemize}
    \item \textbf{Frontend API errors}: Ensure the running ports are correct(5000 for backend and 5173 for frontend).
    \item \textbf{Dependency installation issues}: Make sure the correct versions of .NET SDK, Node.js, and npm are installed.
\end{itemize}


\chapter{Effort spent}
\section{Effort Spent}

\begin{table}[ht!]
\centering
\renewcommand{\arraystretch}{1.6}
\begin{tabular}{|c|l|c|}
\hline
\textbf{Member of Group} & \textbf{Effort Spent} & \textbf{Hours} \\ \hline
\multirow{5}{*}{Giovanni Vaccarino} 
    & Introduction & 2h \\ \cline{2-3}
    & Overall Description & 1h \\ \cline{2-3}
    & Specific Requirements & 3h \\ \cline{2-3}
    & Formal Analysis & 1h \\ \hline
\multirow{5}{*}{Vittorio Palladino} 
    & Introduction & 2h \\ \cline{2-3}
    & Overall Description & 1h \\ \cline{2-3}
    & Specific Requirements & 1h \\ \cline{2-3}
    & Formal Analysis & 4h \\ \hline
\multirow{5}{*}{Nicolò Vacis} 
    & Introduction & 1h \\ \cline{2-3}
    & Overall Description & 3h \\ \cline{2-3}
    & Specific Requirements & 2h \\ \cline{2-3}
    & Formal Analysis & 1h \\ \hline
\end{tabular}
\caption{Effort spent by each member of the group}
\end{table}


\chapter{References}
\section{References}
\begin{itemize}
    \item 2024-2025 Software Engineering 2 - Assignment RASD
\end{itemize}


\cleardoublepage

\end{document}
