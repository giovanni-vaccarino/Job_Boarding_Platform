In this section, we outline the implementation strategy, the integration methodology, and the comprehensive testing plan for the job boarding web application, developed using React for the frontend and ASP.NET Core as a monolithic server for the backend. The architecture chosen ensures high decoupling of code, primarily facilitated by dependency injection.

\subsection{Development Plan}

The development process will follow a bottom-up approach, focusing on the incremental building of backend components and simultaneous development of the frontend. The backend is structured into multiple sections, each serving a distinct purpose and responsible for various features. The key features to be developed are as follows:

\begin{itemize}
    \item \textbf{Authentication Management}: Manages user registration, login, and session handling. It is foundational, as most other features depend on authenticated access.
    \item \textbf{Student Information Management}: Handles CRUD operations for student profiles, including their personal details, resumes, and application history.
    \item \textbf{Company Information Management}: Supports company profiles
    \item \textbf{Internship Management}: Manages internship postings, applications, and status updates.
    \item \textbf{Matching Management}: Manages matching processes and recommendation improving system.
    \item \textbf{Feedback Management}: Manages the feedback system that guarantees students and companies to provide feedbacks on the platform and on companies(for students) and students(for companies)  .
\end{itemize}

Each feature will be developed in sequence, respecting dependencies (e.g., the Authentication Module must be implemented before Student and Company Information Management). Development will be conducted alongside unit testing for each component to ensure correctness and stability. Additionally, testing of the backend functionalities is performed using Postman to validate API endpoints, ensure expected behavior, and facilitate the later integration with the frontend components.

\subsection{Component Integration and Testing}

Once individual components are developed and unit tested, the integration phase begins. Integration will be performed in stages:

\begin{itemize}
    \item \textbf{Phase 1: Backend Integration}: Integrate and communicate between the several backend sections developed.
    \item \textbf{Phase 2: Backend-Frontend Integration}: Connect the frontend React application to the backend API endpoints. This step involves verifying that data requests and responses between the client and server are functioning as intended.
\end{itemize}

During this phase, integration tests will be conducted to verify that the combined components function as expected when communicating with each other.

\subsection{System and End-to-End (E2E) Testing}

Once all components have been successfully integrated, system testing will be conducted to ensure that the complete solution functions cohesively. This phase focuses on verifying that all integrated components interact correctly and perform as expected in a unified environment. System testing will comprehend the following type of tests:

\begin{itemize}
\item \textbf{Load Testing}: Assess the system's ability to handle varying levels of user traffic and operational demands, ensuring stability under stress. \item \textbf{Performance Testing}: Measure the system's response times and performance metrics under both typical and peak usage conditions to ensure reliability and scalability.
\end{itemize}

End-to-End (E2E) testing will simulate real user interactions with the application through the frontend interface, validating the application's functionality, workflow, and overall behavior from start to finish.



\subsection{Acceptance Testing}

The final phase of testing is acceptance testing, which involves stakeholders and representative users. This testing stage is crucial to ensure that the system meets the business requirements and user expectations. Once a beta version of the software is available, the acceptance testing process will proceed as follows:

\begin{itemize}
    \item \textbf{User-Based Scenarios}: Real users, including stakeholders, will perform typical tasks on the system (e.g., creating an account, browsing job postings, and applying for internships).
    \item \textbf{Feedback Collection}: User feedback will be gathered to identify usability issues, missing features, or areas for improvement.
    \item \textbf{Validation}: Confirm that the system meets the acceptance criteria defined in the project requirements.
\end{itemize}

Acceptance testing will conclude with a review meeting, where results and feedback are discussed, leading to necessary adjustments before the final release.
